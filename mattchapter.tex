\chapter{On Differential Equations in Molecular Virology}

\section{Introduction}

\subsection{Biological Background of Molecular Virology}

Before diving into the maths of virology, we will briefly look at some of the biological processes involved, as a way of giving context to the maths that is employed later. Viral Pathogenesis is a very complex process with lots of varying factors. It can be somewhat boiled down to a balancing act between the host and the virus. The goal of any virus that wants to survive is to hijack the cells in a host and replicate as much as possible. This is done by a process known as \textit{replication}.\footnote{Creative, I know} $\cite{cann}$

There are three stages involved in replication, \textbf{Initiation of Infection}, \textbf{Replication and Expression of the viral genome} and  \textbf{Release of mature virions}. The modelling we will undertake primarily focuses on steps 1 and 3. 

\subsection{Overview Differential Equations in Molecular Virology}

Differential Equations are very relevant in applied mathematics. Virology is no exception. In this chapter we will be focusing on a micro scale, looking in depth at the equations used to model in-host viral dynamics. There are two main types of differential equations we will look at: ordinary (ODE) and stochastic (SDE). 
We assume the reader is familiar with ODEs, but SDE's are slightly more specialised. $\cite{sdeProc}$.

\begin{definition}
    A \textbf{Stochastic Differential Equation (SDE)} is a differential equation of the form 
    \[
        dX_t = b(X_t,t)dt + \sigma (X_t,t)dW_t
    \]  
\end{definition}

We see two terms here that need further explanation.

\begin{itemize}
    \item $b(X_t,t)$ is the \textbf{drift coefficient} of the equation. It describes the deterministic part of the equation. \\
    \item $\sigma (X_t,t)dW_t$ is the \textbf{diffusion coefficient}. It describes random motion, which is what seperates SDEs from ODEs. If this term was not here, we would simply have an ODE.
\end{itemize}

\section{An ODE model for in-host viral dynamics}

\subsection{Deriving the model}

Consider a human being. We can split the molecular constituents of this person into 3 categories: \textbf{Uninfected cells}, \textbf{Cells already infected cells} and \textbf{Free virions}. We denote the amount of these present at a time $t$ as  $T(t)$,  $I(t)$ and  $V(t)$ respecitvely.
The body of the person in question will produce fresh uninfected cells at a rate $\lambda$, however these cells will be removed from this category if they die or become infected. We use  $k$ to denote the rate at which uninfected cells become infected per virion. Secondly, these cells die naturally at a rate of  $\mu_T$. In the same way, $\mu_I$ is the rate at which infected cells are removed, the three options for this are natural cell death, immune response and \textit{lysis}, which is where fresh virions are ejected from the cell, breaking the membrane and killing the cell.
Similarly, $\mu_V$ is the rate at whcic virions are removed from the host. Finally, $N$ gives the average number of new virions produced during the life of an infected cell. 

With all of this in mind, we can derive differential equations for $T$,  $I$, and  $V$. Starting with $T$, we have a constant  $\lambda$ feeding new cells, but then we are removing them at a rate $k$ per virion, and a proportion $\mu_T$ die of natural cell death. So we have:

\[
    \frac{dT}{dt} = \lambda - \mu_TT - kTV    
\] 

Then for $I$, the $kTV$ term from the above becomes positive, because these cells are going from uninfected to infected. These cells also die off at a rate of $\mu_T$ which encompasses natural death and the immune system. Putting this together gives:

\[
    \frac{dI}{dt} = kTV - \mu_II    
\] 

Finally, for $V$, we have to include N and $\mu_I$ into one term, because  $N$ virions can only be ejected from a cell if it dies. THen again we have the term  $\mu_V$ for the removal of these cells from the host.

\[
    \frac{dV}{dt} = N\mu_II - \mu_VV
\] 

All that is left to do is combine these three into a sytem of ODEs which we call the \textbf{Standard Model for In-Host Viral Dynamics} $\cite{pan}$.

\begin{equation}
    \begin{cases}
        \frac{dT}{dt} = \lambda - \mu_TT - kTV \\
        \frac{dI}{dt} = kTV - \mu_II \\
        \frac{dV}{dt} = N\mu_II - \mu_VV
    \end{cases}
\end{equation}

\subsection{Model of an Unifected Person}

Before someone is infected, they have no free virions. With $V = N = I = 0$, we now model the person with two equations.

\begin{equation}
        \frac{dT}{dt} &= \lambda - \mu_TT \\
\end{equation}

The solution for this system is given as 

\begin{equation}
        T(t) &= Ae^{-\mu_T t}+\frac{\lambda}{\mu_T} \\
\end{equation}

Since there are no infected cells, and no free virions, we just have the number of target cells increasing exponentially. Which makes sense because one's body creates more cells than it kills off- or at least, ideally that's what happens.

We can plot a sketch of what this looks like with values $\mu_T = \frac{1}{10}* \lambda$. This can be seen in figure 1. Clearly the values are restircted by my computing power, a human person (hopefully) has more than a total of 20000 cells.

\begin{figure}[t]
    \centering
    \includegraphics[scale=0.8]{../figures/Figure1.png}
    \caption{Sketch of T(t)}
\end{figure}

\newpage

So we see that the exponential shape as mentioned. This is clearly slightly simplified, to make a more realistic model, we wouldn't have $\frac{d \lambda}{dt} = 0$. As a person gets older, their body will produce less cells, which would cause the graph to curve off and eventually stop. 

\subsection{Model for an Infected Person}
We now turn to someone who has caught a virus. There are now $V$ free virions present. The important factor is N. To see just how much this factor varies the results, a plot of this system was created in the \textit{Python} programming language.\footnote{Python was selected due to it's efficiecny and helpful modules which makes solving these equations much easier.}

We first look at positive N. When N is 1, each time an infected cell dies, only 1 new virion is released, analagous for N=10 and 50. The plotted graphs for these models can be seen below:

\begin{figure}[h]
    \centering
    \begin{minipage}[b]{0.4\textwidth}
        \includegraphics[width=\textwidth]{../figures/N1.png}
        \caption{ODE Model with N=1}%
    \end{minipage}
    \hfill
    \centering
    \begin{minipage}[b] {0.4\textwidth}   
        \includegraphics[width=\textwidth]{../figures/N10.png}
        \caption{ODE Model with N=10}%
    \end{minipage}
\end{figure}

\begin{figure}[t]
    \centering
    \includegraphics[width=0.5\linewidth]{../figures/N50.png}
    \caption{ODE Model with N=50}%
\end{figure}

We can see that these equations result in a oscilating graph in each case. These coupled differential equations show how the amount of infected vs target cells changes over a given time period. When N is lower, the initial oscillation is greater. From a biological standpoint, this makes perfect sense as the host's body won't be as agressive in attacking the pathogen. \footnote{This is analagous to the fable of the boiling frog.} In the scenario where N is lower, we can see from the above models that the infection is sustained for longer, whereas when N is longer, the host will be more aggressive in fighting the pathogen, and so the infection won't last as long. Through genetic variations, the virus aims to find a balancing act between these performances. If a virus kills off its host too quickly, it will not reproduce and spread to others, which goes agains the aim of evolution.

Further, note that here we have not considered medications and treatments, this will be discussed in a later chapter.

\subsection{Advancing the ODE model}

Differential equations are a great tool for this kind of scientific enquiry, and the model we have seen above has demonstrated it's power in acurately describing reality. But how can we improve it? Firstly, it depends on the virus being modelled. There are many viral infections, HIV, Dengue, SARS-CoV2 to name a few. 

%parameter estimation!

\bibliography{refs}{}
\bibliographystyle{plain}
